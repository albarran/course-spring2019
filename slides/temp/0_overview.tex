\documentclass[aspectratio=169]{beamer}
%\documentclass[11pt]{beamer}
%\documentclass[handout]{beamer}
\beamertemplatenavigationsymbolsempty



\usefonttheme{serif}
\usefonttheme{professionalfonts} % using non standard fonts for beamer
\usepackage{mathpazo}
\usepackage[scaled=.95]{helvet}% uncomment these if required

\usepackage{etex}
\reserveinserts{28}

\usepackage{fontawesome}
\usepackage{courier}
\usepackage{amsmath}
\usepackage{amssymb}
\usepackage{verbatim}
\usepackage{tikz}
\usepackage{graphicx}
\usepackage{epstopdf}
\usepackage{color}
\usepackage{psfrag}
\usepackage{hyperref}
\usepackage{pgfpages}
\usepackage{subfig}
\usepackage{booktabs}
\usepackage{appendixnumberbeamer}
\usepackage{econometrics}

\setbeamertemplate{frametitle}
{
\begin{center}
\textsc{{\large \insertframetitle}}
\end{center}
}


\definecolor{UniBlue}{RGB}{83,121,170}
\definecolor{UniGreen}{RGB}{00,128,00}

%% Theorems
\newcommand{\Exp}{\mathrm{E}}
\newcommand{\Var}{\mathrm{Var}}
\newcommand{\Cov}{\mathrm{Cov}}
\newcommand{\Corr}{\mathrm{Corr}}
\newcommand{\ts}{\textsuperscript}

\setbeamercolor{alerted text}{fg=UniGreen} 
\setbeamerfont{alerted text}{series=\bfseries, shape=\itshape} 

\addtobeamertemplate{navigation symbols}{}{%
    \usebeamerfont{footline}%
    \usebeamercolor[fg]{footline}%
    \hspace{1em}%
    \insertframenumber
}
\setbeamerfont{footline}{size=\footnotesize, series=\bfseries}
\setbeamercolor{background box}{bg=lightgray}

\newcommand\YUGE{\fontsize{48}{60}\selectfont}

\title{{\large Course Overview}}
\date{April 26, 2018}
\author{Itamar Caspi}
\institute{ml4e @ HUJI, 2018}


% PROGRESS BAR =====================================================

% PROGRESS BAR 
\usetikzlibrary{calc}

\definecolor{pbblue}{RGB}{52,51,179}% filling color for the progress bar
\definecolor{pbgray}{RGB}{255,255,255}% background color for the progress bar


\makeatletter
\def\progressbar@progressbar{} % the progress bar
\newcount\progressbar@tmpcounta% auxiliary counter
\newcount\progressbar@tmpcountb% auxiliary counter
\newdimen\progressbar@pbht %progressbar height
\newdimen\progressbar@pbwd %progressbar width
\newdimen\progressbar@tmpdim % auxiliary dimension

\progressbar@pbwd=\paperwidth
\progressbar@pbht=5ex

% the progress bar
\def\progressbar@progressbar{%
    \progressbar@tmpcounta= \insertframenumber % max = ?
    \progressbar@tmpcountb=\inserttotalframenumber      
    \progressbar@tmpdim=.5\progressbar@pbwd
    \multiply\progressbar@tmpdim by \progressbar@tmpcounta
    \divide\progressbar@tmpdim by \progressbar@tmpcountb
    \progressbar@tmpdim=2\progressbar@tmpdim
  

\begin{tikzpicture}[very thin]
    \shade[top color=pbgray,bottom color=pbgray,middle color=pbgray]
      (0pt, 0pt) rectangle ++ (\progressbar@pbwd, \progressbar@pbht);

      \shade[draw=pbblue,top color=pbblue,bottom color=pbblue,middle color=pbblue] %
        (0pt, 0pt) rectangle ++ (\progressbar@tmpdim, \progressbar@pbht);

  \end{tikzpicture}%
}
\addtobeamertemplate{headline}{}
{%
  \begin{beamercolorbox}[wd=\paperwidth,ht=1ex,center,dp=0ex]{white}%
    \progressbar@progressbar%
  \end{beamercolorbox}%
}
\makeatother

% END PROGRESS BAR =====================================================

\begin{document}
\maketitle

\begin{frame}{Logistics}
Course Google Drive:\\
\url{https://drive.google.com/open?id=1IiZroSBQOV4I6bYafVehLrvXZ75vICXc}
\begin{itemize}
\item Syllabus
\item Slides
\item Textbooks and background papers
\item R tutorial and code
\item Assignment
\end{itemize}
\end{frame}

\begin{frame}{Assignment}
\begin{itemize}
\item Your course assignment is a Kaggle competition:
\url{https://www.kaggle.com/t/38a084622e714eee98b70c7574781060}
\item where you will be required to predict the median value of a house in Boston districts using one or more of the machine learning algorithms you will learn in class
\end{itemize}
\end{frame}

\begin{frame}{People}
Instructors involved in providing this course:
\begin{itemize}
\item Ariel Mansura
\item Shir Kamenetsky
\item Itamar Caspi
\item Igor Rochlin (GSTAT, \url{igorochlin@gmail.com} )
\end{itemize}
\end{frame}

\begin{frame}{Prerequisites}
\begin{itemize}
\item Advanced level (M.A.) course in Statistics/Econometrics
\item Working knowledge of R
\end{itemize}

\end{frame}

\begin{frame}{Resources}
Main texts:
\begin{itemize}
\item \textit{An Introduction to Statistical Learning, with Applications in R} (ISLR) by James, Witten, Hastie and Tibshirani (Springer, 2013)
\item \textit{The Elements of Statistical Learning - Data Mining, Inference, and prediction} (ESL) by Friedman, Tibshirani, and Hastie (Springer, 2008)
\end{itemize}
\end{frame}

\begin{frame}{Course Goals}
What this course is about
\begin{itemize}
\item learn tools that will enable you to work with big data of the type you are familiar with
\item learn how to implement these tools using R and
\begin{itemize}
\item produce quality prediction
\item classify unstructured data
\end{itemize}
\end{itemize}
What this course is NOT about
\begin{itemize}
\item state-of-the-art machine learning methods (gradient boosting, deep learning, etc.)
\item causal inference
\item efficient computation
\item data querying (SQL)
\item complex data structures
\end{itemize}
\end{frame}

\begin{frame}{Syllabus}
Here is a list of the main topics we intend to cover during the course:
\begin{itemize}
\item Basic concepts (Itamar)
\item Regression and $ K $-nearest neighbors (Ariel)
\item Classification methods (Igor)
\item Support vector machines (SVM) (Igor)
\item Neural networks (Igor)
\item LASSO, Ridge and principal component regression (Ariel)
\item Unsupervised learning (Igor)
\end{itemize}
Bonus lecture: \textbf{"Machine learning applications in the Israeli fintech industry"} (tentative title) by Ido Mintz, lead data scientist @ Intuit.
\end{frame}

\begin{frame}{Feedback}
Your feedback is important! Please feel free to share with us your comments, concerns and suggestions on the course in person, email or anonymously here:
\url{http://www.admonymous.com/boibigdata}
\end{frame}



\end{document}